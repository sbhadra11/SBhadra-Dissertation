\section{Conclusions}
\label{sec:dbb:conclusion}

We see that determination of the COR of an alkaline battery through a simple bounce test provides an accurate measure of the morphological state of the anode. Based on comparison, the bounce test is capable of determining the percolation of ZnO in the anode to within 13\% of the EDXRD determined value. As such, the bounce test functions as a measure of the bulk properties of a battery. After deductive analysis of the structural changes within the LR6 zinc alkaline battery, we see that the closest correlation with the beginning of the increase in COR in the consumption of water to form ZnO at the anode.  As demonstrated, dehydration of the anode without ZnO network formation does not cause the battery to bounce more.  However, ZnO does not form if there is no water to be consumed, so we are left to say that the bounce is most likely caused within the standard discharge mechanism by 1) the formation of ZnO by the consumption of water and 2) the point at which enough water has been consumed to form a percolated ZnO network.  It must be noted that there is still enough electrolyte to maintain low cell impedance over the entire depth of discharge: more ZnO is forming than water being lost, as a percentage of the amount of ZnO and water to begin with (roughly $<$ 0.1 g and  2.5 g, respectively).

The battery most likely begins to bounce because of displacement of water by solid ZnO bridges between particles of zinc in the gel.  These bridges provide less impeding and attenuating paths for pressure waves, in turn making the battery “bouncier.”  The sudden onset of increased COR between 400 and 600 mAhr correlates strongly with EDXRD evidence of ZnO percolation, and the COR first quickly rises and then gradually tapers off near 0.66 after 1500 mAh have passed.  The leveling of the COR correlates with the point at which the formation of ZnO crosses from Type I dominated to Type II dominated.

