\section{Conclusions}
\label{sec:alkbw:conclusion}

Despite their ubiquity, there is still much to be learned about the mechanical properties of alkaline AA batteries. In this work, we show that using ultrasonic pulses to perform electrochemical acoustic time-of-flight measurements can provide a simple route to determining the dynamic mechanical properties of such batteries, especially those of the Zn gel anode. When paired with additional resources like energy dispersive x-ray diffraction or scanning electron microscopy, we can draw correlations that greatly increase the utility of our method. By tracking the number of echoes in the ToF signal during discharge, we can estimate the state of the anode in terms of oxidation and dehydration. Tracking total transmitted signal amplitude provides similar data as we can correlate strong signal amplitude to high degrees of densification and dehydration, and the rate of change of the signal amplitude can provide information about the rate of ZnO formation. Finally, tracking the ToF peak positions can offer insight into the composition of the anode, as EDXRD data shows that shifts to higher ToF values often correspond to the presence of ZnO, while shifts to lower ToF values occur more readily with the appearance of higher order ZnO EDXRD peaks that correlate well with anode densification. ToF peak tracking also gives insight into the  uniformity of the oxidizing anode as fluctuations in ToF peak position appear during periods of increased oxidation. EAToF measurements may even provide information about the relative particle loading and moisture content of the electrode components, as smaller particle sizes will pack more densely and give a stronger transmission signal, while more moisture will more strongly attenuate the transmission signal. With this knowledge we can not only determine information about the state of charge and state of health of alkaline batteries, but also begin to quantify the differences between different brands of batteries \textit{in operando} and therefore be more informed in our selections of electrochemical energy storage.