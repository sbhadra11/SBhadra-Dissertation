Batteries have become a ubiquitous form of electrochemical energy storage, but thus far the methods for measuring the mechanical properties of batteries and their component materials \textit{in operando} have lagged far behind the methods for measuring the corresponding electrical properties. In this thesis, we demonstrate methods for determining the changes in materials properties of an electrochemical energy storage cell both \textit{ex situ} and \textit{in operando}.

We begin by establishing the impact of micro-scale morphology changes on the macro-scale dynamic mechanical response in commercial alkaline AA cells. Using a bounce test, the coefficient of restitution (COR) of the cell is shown to increase non-linearly as a function of state of charge (SOC). We show that the reason for the increase in the COR stems from the spatially-dependent oxidation of the Zn anode, with an initial increase corresponding to the formation of a percolation pathway of ZnO-clad Zn particles spanning the radius of the anode. The subsequent saturation of the COR is shown to result from the ultimate densification and desiccation of the Zn anode.

Building from this, we present a generalized \textit{in operando} solution for materials characterization in batteries using ultrasonic interrogation. The materials properties of battery components change during charge and discharge, resulting in a change in the sound speed of the materials. By attaching transducers on either side of a battery during cycling and sending ultrasonic pulses through each cell we observe the changes in the time of flight (ToF) of the pulses, both in reflection and transmission. We show that the changes in ToF correspond to both SOC and SOH in a variety of battery chemistries and geometries, and detail a corresponding acoustic conservation law model framework. 

Finally, we perform these electrochemical acoustic time of flight (EAToF) experiments on commercial alkaline AA cells. By correlating the results with energy dispersive x-ray diffraction (EDXRD) data and previous bounce test data, we show that EAToF is capable of determining the morphology changes in the anode and cathode. We also show that using EAToF, the materials quality differences between multiple AA battery brands can be determined.