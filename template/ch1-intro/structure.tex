\section{Dissertation structure}
\label{sec:intro:structure}

This work investigates how mechanical interrogation and characterization methods can be implemented to develop an \textit{in operando} method of materials analysis in batteries. Chapter~\ref{ch:dbb} sets the groundwork for this by analyzing \textit{ex situ} an interesting phenomenon in alkaline AA batteries where a cell's bounce height after being dropped on end onto a rigid surface will depend on its state of charge. The results of this study show that inhomogeneities in the oxidation of the Zn anode manifest macroscopically as a change in the coefficient of restitution of the cell. Building from this, Chapter~\ref{ch:bw} demonstrates an \textit{in operando} method for analyzing the density and elastic modulus of the components of a battery regardless of chemistry or geometry through the use of ultrasonic interrogation. The study described in this chapter help to establish an acoustic "fingerprint" for various batteries, and show how acoustics can be used to ascertain a battery's state of charge and state of health. Finally, Chapter~\ref{ch:alkbw} employs this ultrasonic interrogation technique to alkaline AA batteries to show how the oxidation of the anode can be monitored \textit{in operando}, and how this information can be used to quantify differences between multiple brands of alkaline AA batteries beyond just price and name. Thus, this work lays the foundation for a new method of mechanical characterization of batteries that extends our knowledge beyond just electrical phenomena and testing.   