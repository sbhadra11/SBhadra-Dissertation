\section{Motivation}
\label{sec:intro:motivation}

With the explosion in popularity of portable consumer electronics, as well as the burgeoning field of electric vehicles, batteries have been thrust into the forefront of energy storage research. Due to this sudden interest much work is being undertaken to improve the capacity and performance of both primary (single-use) and secondary (rechargeable) batteries at multiple size scales, from individual batteries for portable devices, to electric vehicles, and even grid storage applications.

Secondary batteries all experience varying degrees of capacity fade, and all batteries experience eventual failure. The mechanisms behind these are understood to varying degrees for different battery chemistries, but to date there are few methods for monitoring state of charge and state of health \textit{in operando}.~\cite{Pop2005-qv} The majority of these methods rely upon "book-keeping," in which the amount of charge passed is noted and correlated to voltage profiles during charging and discharging, with a smaller subset relying on electrochemical methods such as impedance measurements.~\cite{Huet_undated-ik} While these methods are used widely, they do not allow insight into the materials phenomena occurring within the battery. It is clear that better monitoring of battery state of charge and state of health requires thinking outside of the electrical box by finding ways to monitor the materials properties of battery components during use. 

