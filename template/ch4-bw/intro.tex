\section{Introduction}
\label{sec:bw:intro}

The very quality that makes batteries interesting for academic research is concurrently a source of frustration for their practical implementation: each chemistry has a specific physical fingerprint, which leads to unique cycling behaviors, desired or otherwise. The standard suite of electrochemical tools provides a window into the physical changes of each chemistry, but it is at best an abstract representation of the physical changes occurring in the cell. \textit{In situ} and \textit{in operando} optical~\cite{Ito2011-ir,Gallaway2010-xk}, electron~\cite{Santhanagopalan2014-xh,Karki2014-jj,Radisic2006-zq}, x-ray[6–9] and neutron scattering[10–12] methods have provided rich data which describe the behavior of idealized cells, but with few exceptions[13–16] it has been difficult to directly probe the physical changes in conventional batteries. This is a true detriment to the field, as scaling up cells is not a trivial linear exercise: physical insights into large-scale cells, without the need for expensive equipment such as a synchrotron light source, would be welcome both in academia and in industry.

We present the framework for a non-invasive, \textit{in operando} method that is able to extract a rich data set from numerous battery designs by exploiting a physical truth that underlies all closed electrochemical systems: they are, by design, reactors which redistribute density as a function of state of charge (SOC) in the ideal case and, additionally, as a function of state of health (SOH) in reality. Regardless of the reaction mechanism (intercalation, dissolution/reprecipitation, phase change, etc.), the density and elastic modulus of an electrode changes as a function of its SOC, and this distribution as well as the rate of change of this distribution can act as a fingerprint of SOH. For what we believe is the first time in the literature, in the present study we use acoustic ultrasonic transducers to probe the changes in density distribution in real time, and provide a model which describes how ultrasonic echoes within an arbitrary cell change as a function of the SOC. The concept for this approach is illustrated for an example cell in Fig. 1. In this article we discuss a simple method that may be used with one or two transducers to characterize SOC and SOH. Beyond correlations between density and acoustic signal amplitude, acoustic attenuation will also change as a function of the effective modulus of each layer in the battery.

Recently, there has been significant work correlating static strain at a macroscopic level to SOC and SOH in batteries which exhibit significant volume changes.17–21 In some cases, the macroscopic strain can be exploited as an actuator.22 Acoustics have been employed to detect emissions from macroscopic cracking,23–26 and microscopic AFM measurements27,28 and models have been developed to determine the causes and critical aspects of “electrochemical shock.”29–32 While there have been efforts in ultrasonic imaging of full cells, they have focused on the examination of irreversible failure through delamination and cracking.33–35 To the best of our knowledge, no efforts have correlated slight-to-moderate mechanical degradation or SOC with ultrasonic interrogation. In this work, we employ acoustic methods that were developed for flaw detection of bulk metals and welds36–38 to accurately correlate state of charge within a battery to subtle changes within materials and between layers.