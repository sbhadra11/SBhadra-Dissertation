\documentclass[]{article}
\usepackage{lmodern}
\usepackage{amssymb,amsmath}
\usepackage{ifxetex,ifluatex}
\usepackage{fixltx2e} % provides \textsubscript
\ifnum 0\ifxetex 1\fi\ifluatex 1\fi=0 % if pdftex
  \usepackage[T1]{fontenc}
  \usepackage[utf8]{inputenc}
\else % if luatex or xelatex
  \ifxetex
    \usepackage{mathspec}
    \usepackage{xltxtra,xunicode}
  \else
    \usepackage{fontspec}
  \fi
  \defaultfontfeatures{Mapping=tex-text,Scale=MatchLowercase}
  \newcommand{\euro}{€}
\fi
% use upquote if available, for straight quotes in verbatim environments
\IfFileExists{upquote.sty}{\usepackage{upquote}}{}
% use microtype if available
\IfFileExists{microtype.sty}{%
\usepackage{microtype}
\UseMicrotypeSet[protrusion]{basicmath} % disable protrusion for tt fonts
}{}
\ifxetex
  \usepackage[setpagesize=false, % page size defined by xetex
              unicode=false, % unicode breaks when used with xetex
              xetex]{hyperref}
\else
  \usepackage[unicode=true]{hyperref}
\fi
\hypersetup{breaklinks=true,
            bookmarks=true,
            pdfauthor={},
            pdftitle={},
            colorlinks=true,
            citecolor=blue,
            urlcolor=blue,
            linkcolor=magenta,
            pdfborder={0 0 0}}
\urlstyle{same}  % don't use monospace font for urls
\usepackage{longtable,booktabs}
\setlength{\parindent}{0pt}
\setlength{\parskip}{6pt plus 2pt minus 1pt}
\setlength{\emergencystretch}{3em}  % prevent overfull lines
\providecommand{\tightlist}{%
  \setlength{\itemsep}{0pt}\setlength{\parskip}{0pt}}
\setcounter{secnumdepth}{0}

\date{}

% Redefines (sub)paragraphs to behave more like sections
\ifx\paragraph\undefined\else
\let\oldparagraph\paragraph
\renewcommand{\paragraph}[1]{\oldparagraph{#1}\mbox{}}
\fi
\ifx\subparagraph\undefined\else
\let\oldsubparagraph\subparagraph
\renewcommand{\subparagraph}[1]{\oldsubparagraph{#1}\mbox{}}
\fi

\begin{document}

\textbf{Cite this: DOI: 10.1039/x0xx00000x}

Received 00th January 2012,

Accepted 00th January 2012

DOI: 10.1039/x0xx00000x

\textbf{www.rsc.org/}

{\\
 }\textbf{Electrochemical-Acoustic Time of Flight: \emph{In Operando}
Correlation of Physical Dynamics with Battery Charge and Health}

A.G. Hsieh,\textsuperscript{a,c} S. Bhadra,\textsuperscript{b,c} B.J.
Hertzberg,\textsuperscript{a,c} P.J. Gjeltema,\textsuperscript{a} A.
Goy,\textsuperscript{b} J.W. Fleischer\textsuperscript{b}
\textsuperscript{~}and D.A. Steingart\textsuperscript{a,c*}

We demonstrate that a simple acoustic time-of-flight experiment can
measure the state of charge and state of health of almost any closed
battery. An acoustic conservation law model describing the state of
charge of a standard battery is proposed, and experimental acoustic
results verify the simulated trends; furthermore, a framework relating
changes in sound speed, via density and modulus changes, to state of
charge and state of health within a battery is discussed. Regardless of
the chemistry, the distribution of density within a battery must change
as a function of state of charge and, along with density, the bulk
modulus of the anode and cathode changes as well. The shifts in density
and modulus also change the acoustic attenuation in a battery.
Experimental results indicating both state-of-charge determination and
irreversible physical changes are presented for two of the most
ubiquitous batteries in the world, the lithium-ion 18650 and the
alkaline LR6 (AA). Overall, a one- or two-point acoustic measurement can
be related to the interaction of a pressure wave at multiple discrete
interfaces within a battery, which in turn provides insights into state
of charge, state of health, and mechanical evolution/degradation.

Introduction

{ }The very quality that makes batteries interesting for academic
research is concurrently a source of frustration for their practical
implementation: each chemistry has a specific physical fingerprint,
which leads to unique cycling behaviors, desired or otherwise. The
standard suite of electrochemical tools provides a window into the
physical changes of each chemistry, but it is at best an abstract
representation of the physical changes occurring in the cell\emph{. In
situ} and \emph{in operando} optical\textsuperscript{1,2},
electron\textsuperscript{3--5}, x-ray\textsuperscript{6--9} and neutron
scattering\textsuperscript{10--12} methods have provided rich data which
describe the behavior of idealized cells, but with few
exceptions\textsuperscript{13--16} it has been difficult to directly
probe the physical changes in conventional batteries. This is a true
detriment to the field, as scaling up cells is not a trivial linear
exercise: physical insights into large-scale cells, without the need for
expensive equipment such as a synchrotron light source, would be welcome
both in academia and in industry.

{ }We present the framework for a non-invasive, \emph{in operando}
method that is able to extract a rich data set from numerous battery
designs by exploiting a physical truth that underlies all closed
electrochemical systems: they are, by design, reactors which
redistribute density as a function of state of charge (SOC) in the ideal
case and, additionally, as a function of state of health (SOH) in
reality. Regardless of the reaction mechanism (intercalation,
dissolution/reprecipitation, phase change, etc.), the density and
elastic modulus of an electrode changes as a function of its SOC, and
this distribution as well as the rate of change of this distribution can
act as a fingerprint of SOH. For what we believe is the first time in
the literature, in the present study we use acoustic ultrasonic
transducers to probe the changes in density distribution in real time,
and provide a model which describes how ultrasonic echoes within an
arbitrary cell change as a function of the SOC. The concept for this
approach is illustrated for an example cell in Fig. 1. In this article
we discuss a simple method that may be used with one or two transducers
to characterize SOC and SOH. Beyond correlations between density and
acoustic signal amplitude, acoustic attenuation will also change as a
function of the effective modulus of each layer in the battery.

\textbf{Figure 1 \textbar{} Ultrasonic interrogation of a representative
battery. a.} Schematic representation of the experimental/modeling
configuration, showing a typical battery with the various packaging,
current collector, electrode, and separator layers as well as the two
acoustic transducers (pulse/listen and listen) used for ultrasonic
interrogation. \textbf{b.} Example illustration of the increase in time
of flight (ToF) of the transmitted signal as a function of SOC that
occurs during discharge; this shift is a result of the changes in
electrode densities as the SOC (i.e., Li content, \emph{x}) changes.

{ }Recently, there has been significant work correlating static strain
at a macroscopic level to SOC and SOH in batteries which exhibit
significant volume changes.\textsuperscript{17--21} In some cases, the
macroscopic strain can be exploited as an actuator.\textsuperscript{22}
Acoustics have been employed to detect emissions from macroscopic
cracking,\textsuperscript{23--26} and microscopic AFM
measurements\textsuperscript{27,28} \textsubscript{} and models have
been developed to determine the causes and critical aspects of
``electrochemical shock.''\textsuperscript{29--32} While there have been
efforts in ultrasonic imaging of full cells, they have focused on the
examination of irreversible failure through delamination and
cracking.\textsuperscript{33--35} To the best of our knowledge, no
efforts have correlated slight-to-moderate mechanical degradation or SOC
with ultrasonic interrogation. In this work, we employ acoustic methods
that were developed for flaw detection of bulk metals and
welds\textsuperscript{36--38} to accurately correlate state of charge
within a battery to subtle changes within materials and between layers.

Methods

Computational

{ }A standard 1D acoustic conservational law model was used to explore
the effect of density and modulus on the echoing behavior of an
ultrasonic pulse, based on the mismatch in sound speed between adjacent
layers within a battery. The speed of sound, \emph{c}, in a solid
material can be calculated from the Newton-Laplace equation:

~~(1)

where \emph{E} is the elastic modulus, and \emph{ρ} is the density. The
Conservation Laws Package (Clawpack)\textsuperscript{39} with the
Riemann acoustics solver was used to simulate a full battery stack (Fig.
1a) using python (via pyclaw). In one dimension, the governing
continuity equations are:

{~}~~(2)

{~~ }~(3)

where \emph{p} is the pressure, \emph{u} is the wave velocity, and the
subscripts \emph{x} and \emph{t} imply spatial and temporal variation,
respectively. Values for the densities and elastic moduli of the
electrode components were assembled from the references in Table 1; the
densities for LiCoO\textsubscript{2} and graphite are listed as a
function of Li content.

\textbf{Table 1 \textbar{}} Densities and elastic moduli of electrode
components for acoustic simulation

\begin{longtable}[c]{@{}llll@{}}
\toprule
\begin{minipage}[t]{0.22\columnwidth}\raggedright\strut
\textbf{Material}
\strut\end{minipage} &
\begin{minipage}[t]{0.22\columnwidth}\raggedright\strut
\textbf{Density}

\textbf{(kg/m}{\textbf{\textsuperscript{3}}}\textbf{)}
\strut\end{minipage} &
\begin{minipage}[t]{0.22\columnwidth}\raggedright\strut
\textbf{Modulus}

\textbf{(GPa)}
\strut\end{minipage} &
\begin{minipage}[t]{0.22\columnwidth}\raggedright\strut
\textbf{Citation}
\strut\end{minipage}\tabularnewline
\begin{minipage}[t]{0.22\columnwidth}\raggedright\strut
LiCoO{\textsubscript{2}}
\strut\end{minipage} &
\begin{minipage}[t]{0.22\columnwidth}\raggedright\strut
4800 - 5150
\strut\end{minipage} &
\begin{minipage}[t]{0.22\columnwidth}\raggedright\strut
168.0
\strut\end{minipage} &
\begin{minipage}[t]{0.22\columnwidth}\raggedright\strut
40
\strut\end{minipage}\tabularnewline
\begin{minipage}[t]{0.22\columnwidth}\raggedright\strut
Graphite
\strut\end{minipage} &
\begin{minipage}[t]{0.22\columnwidth}\raggedright\strut
2260 - 2500
\strut\end{minipage} &
\begin{minipage}[t]{0.22\columnwidth}\raggedright\strut
27.6
\strut\end{minipage} &
\begin{minipage}[t]{0.22\columnwidth}\raggedright\strut
22
\strut\end{minipage}\tabularnewline
\begin{minipage}[t]{0.22\columnwidth}\raggedright\strut
Al
\strut\end{minipage} &
\begin{minipage}[t]{0.22\columnwidth}\raggedright\strut
2700
\strut\end{minipage} &
\begin{minipage}[t]{0.22\columnwidth}\raggedright\strut
69.0
\strut\end{minipage} &
\begin{minipage}[t]{0.22\columnwidth}\raggedright\strut
41
\strut\end{minipage}\tabularnewline
\begin{minipage}[t]{0.22\columnwidth}\raggedright\strut
Cu
\strut\end{minipage} &
\begin{minipage}[t]{0.22\columnwidth}\raggedright\strut
8960
\strut\end{minipage} &
\begin{minipage}[t]{0.22\columnwidth}\raggedright\strut
117.0
\strut\end{minipage} &
\begin{minipage}[t]{0.22\columnwidth}\raggedright\strut
41
\strut\end{minipage}\tabularnewline
\begin{minipage}[t]{0.22\columnwidth}\raggedright\strut
PVDF
\strut\end{minipage} &
\begin{minipage}[t]{0.22\columnwidth}\raggedright\strut
1800
\strut\end{minipage} &
\begin{minipage}[t]{0.22\columnwidth}\raggedright\strut
2.0
\strut\end{minipage} &
\begin{minipage}[t]{0.22\columnwidth}\raggedright\strut
41
\strut\end{minipage}\tabularnewline
\begin{minipage}[t]{0.22\columnwidth}\raggedright\strut
Carbon Black
\strut\end{minipage} &
\begin{minipage}[t]{0.22\columnwidth}\raggedright\strut
1900
\strut\end{minipage} &
\begin{minipage}[t]{0.22\columnwidth}\raggedright\strut
25.0 (est.)
\strut\end{minipage} &
\begin{minipage}[t]{0.22\columnwidth}\raggedright\strut
41
\strut\end{minipage}\tabularnewline
\bottomrule
\end{longtable}

{ }In modeling the electrodes (assuming 80\% active material, 10\%
conductive additive, 10\% binder), to a first approximation the
effective densities of the composite anode and cathode can be estimated
from a mass-fraction weighted sum of the densities of the individual
components. Similarly, the effective moduli of the anode and cathode
layers can be estimated from a volume-fraction weighted sum of the
component moduli. Additionally, a composite can be considered
ultrasonically homogeneous if the particles are much smaller than the
acoustic wavelength, which has been confirmed by both experimental and
theoretical evidence.\textsuperscript{42} For the sake of simplicity,
all of these assumptions were used in our simulations. In future work,
we will examine more complex models of granular materials, such as the
Kelvin-Voight relationship. This is a non-trivial exercise, as the
modeling of acoustic impedance and attenuation in packed-particle beds
is still an active area of research within the acoustic community, with
many questions posed decades ago still
unanswered.\textsuperscript{43,44} This said, our current model captures
a qualitatively meaningful relationship between state of charge and
acoustic behavior.

{ }Changes in the states of charge and lithium content in each electrode
during cycling were determined via a Dualfoil
simulation.\textsuperscript{45} The lithium content for each time step
was extracted from the Dualfoil output file and was then used to
estimate the density changes in each electrode based on values from
studies by Reimers and Dahn.\textsuperscript{40} The changes in density
were then fed into Clawpack to simulate the resulting change in acoustic
behavior as a function of state of charge. As a first approximation, the
modulus was held constant; however, in future studies it may be possible
to determine elastic modulus from the acoustic measurements.

{ }The Dualfoil input files were not modified from the standard
li-ion.in input files created with the package. Within Python, using the
pyclaw library, the input file from the Dualfoil simulation was used to
define the geometry of the cell, and extra layers representing current
collectors and external packaging were added.

Experimental

{ }Electrochemical-acoustic time of flight experiments were performed on
a LiCoO\textsubscript{2}/graphite pouch cell, a cylindrical
Li(NiCoAl)O\textsubscript{2}/graphite 18650 cell, and
Zn/MnO\textsubscript{2} alkaline AA cells (Duracell and CVS Brand).
Specific cell information is provided in the supplemental materials.{~}

{ }An Olympus EPOCH 600 ultrasonic pulser-receiver was used with two
2.25 MHz transducers: one in pulse-echo (reflection) mode and the other
in transmission mode. This allowed the measurement of reflected and
transmitted signals through the cell. A small amount of glycerin was
used to ensure a reliable acoustic interface between the transducers and
the cell, and the transducers were held in place with light pressure
using a custom-designed 3D printed holder (Formlabs Form1+). No
modifications were made to the cells. Custom Python software was written
to control the EPOCH through Pithy.\textsuperscript{46} The applied
pulse was 50 ns long and its transmission/echo behavior was measured out
to 20 µs; measurements were taken every 30 seconds.

{ }To test the cells electrochemically, galvanostatic cycling protocols
were used with a Neware BTS-3000 cycler. For the
LiCoO\textsubscript{2}/graphite and
Li(NiCoAl)O\textsubscript{2}/graphite cells, a C/2.5 cycling rate was
used, with 30 m rest in between each charge and discharge step. For the
alkaline AA cells, single discharge steps at C/20, C/10, and C/5 rates
were used. The Neware was time-synchronized with the EPOCH, and all data
analysis was performed with Python.

Results and Discussion

{ }Supplemental Movie 1 demonstrates a simulated pulse Supplemental
Movie 1 demonstrates a simulated pulse passing through a one-stack cell,
and the resulting reflection and transmission gauge readings. As the
initial pulse passes through each interface, some fraction of the wave
is transmitted and some is reflected, depending on the degree of
mismatch in the sound speed \emph{c} between adjacent layers and whether
\emph{c} increases or decreases from one layer to the next;
additionally, the wave attenuates (i.e., loses energy) as it passes
through the bulk region of each layer.\textsuperscript{47} As each
interface is an opportunity for the pulse to split, as shown in the
movie, the acoustic behavior of the cell quickly becomes complicated as
each new wave interacts not only with interfaces (creating even more
waves) but also with each other. The complex interplay of sound speed
mismatches as well as constructive and destructive interference between
the waves creates the observed reflection and transmission traces.
Furthermore, as the time of flight (ToF) increases, the sound waves
become increasingly dampened due to dissipation and the increasing
number of encountered interfaces. The result is an ``echo chamber''
effect for the longer ToF waves.

{ }To simulate the effect of galvanostatic cycling on the echoing
behavior of the cell, a simple 1D stack of the regions listed in Table 1
was created in the standard cell geometry defined by
Dualfoil.\textsuperscript{45} For a series of time steps, the changes in
electrode densities were calculated as a function of SOC using Dualfoil,
which were then fed into Clawpack to determine the resulting acoustic
behavior of the cell. The snapshots were composited to obtain the
time-resolved simulated acoustic results shown in Fig. 2. This
simulation illustrates that during cycling, there is a measurable shift
in the ToF of the primary transmission gauge reading (indicated by arrow
1) as well as a change in the signal intensity, which are both clearly
functions of the SOC of the battery (i.e., density changes in the anode
and cathode due to lithium intercalation/deintercalation). In addition
to becoming increasingly dampened, the later transmission signals (e.g.,
the trace indicated by arrow 2) display an enhanced shift in ToF between
the charged and discharged states. Similar trends are observed in the
reflection data.{~}

\textbf{Figure 2 \textbar{} Simulation of echo behavior as a function of
SOC.} Clawpack-Dualfoil simulation of the ToF of acoustic echoes
(transmission and reflection modes) in a simple, single 1D battery stack
as a function of SOC; as shown in the scale bar on the right, blue to
yellow red indicates increasing acoustic intensity. Also shown are the
corresponding cell potential and applied current density profiles.
Arrows 1 and 2 are discussed in the text.

{ }To simulate cylindrical and prismatic cell designs, cells consisting
of multiple stacks of electrodes were run through the Clawpack-Dualfoil
simulation, and Fig. S1 demonstrates the effect of cell folding/winding
on the output acoustic signals. We see that as the size of the simulated
cell is increased to 4 electrode stacks, the acoustic behavior becomes
increasingly complex, as there are more and more interfaces for the
waves to pass through, resulting in a progressively increasing number of
waves echoing and interfering with one another. It is worth noting that
commercially available cells generally contain many more layers than
this, which, as we will see further below, results in even more complex
acoustic behaviors. ~

{ }Ultrasonic time of flight analysis was performed on a commercial
LiCoO\textsubscript{2}/graphite pouch cell during galvanostatic cycling:
the reflected and transmitted signals were measured every 30 seconds
during cycling, providing acoustic snapshots of the cell as a function
of time. Time-resolved acoustic results were visualized by compositing
these snapshots into a two-dimensional intensity image of cycle time
against ToF data (selected data shown in Fig. 3a and 3b, full data set
shown in Fig. S2). These composite images demonstrate both the change in
intensity for each acoustic wave received by the transducers as well as
the ToF shift in each wave during cycling. Focusing on the transmission
data (Fig. 3a), we see that each ToF peak shifts towards lower values
and higher intensities during charge, and towards higher values and
lower intensities during discharge. Furthermore, a comparison of
transmission peaks 1, 3, and 10 shows that the later (i.e., longer ToF)
peaks display a more pronounced shift in ToF position between the
charged and discharged states, and are in general less intense. Similar
trends are also observed in the reflection data (Fig. 3b). These
experimental trends are reminiscent of the shifts in ToF and intensity
observed in the Clawpack-Dualfoil simulations. While there are strong
correlations between the simulated SOC-ToF relationship and the overall
experimental ToF trends, due to the assumptions made, our model does not
capture many of the nonlinear intercalation-driven changes that are
known to occur. Nevertheless, we are able to draw several meaningful
correlations between the acoustic and electrochemical results.

{ }For each ToF snapshot, we calculated the total transmitted and
reflected signal amplitudes by summing the intensity across the entire
0-20 µs ToF window, shown in Figure 3c. As the battery is discharged,
acoustic absorption increases (i.e., the transmitted and reflected
intensities decrease), with a notable, but repeatable, exception at the
end of discharge at a cell potential \textless{} 3.5 V, where there is a
dramatic increase in the signal intensities (dotted oval, Fig. 3c). We
believe this is driven primarily by the capacity-limited cathode: near
0\% SOC, LiCoO\textsubscript{2} approaches a hexagonal-to-monoclinic
phase transformation, dramatically altering both the modulus and density
of the cathode.\textsuperscript{40} Following discharge, as the applied
current is removed during the rest step, we see the cell potential
increase as local lithium gradients relax, which in turn relaxes any
build-up in lattice strain due to the aforementioned phase
transformation, and the acoustic signal intensities decrease
accordingly. When the cell is charged, the acoustic intensities decrease
slightly as the phase transformation is reversed, followed by a steady
increase in the intensities with increasing SOC. At the end of charge,
there is a slight, repeatable increase in acoustic absorption when the
cell potential is \textgreater{} 4 V. We believe that this is also
driven primarily by the cathode, as near 100\% SOC, the
LiCoO\textsubscript{2} undergoes a two-phase staging reaction which
changes its density significantly.\textsuperscript{48}{~}

{ }Particularly interesting is the information contained within
individual ToF peak traces. Figures 3d and 3e show the amplitudes of
Transmitted Waves 1 and 3, respectively, and we see that while their
general trends are similar to one another and to the total amplitude,
there are subtle differences between them. Furthermore, amplitude
changes in the individual waves follow different trends with cycle
number. For example, if we focus on the amplitude of Transmitted Wave 1
during the charge step, there is a peak in the 3rd cycle (marked by *1,
Fig. 3d) that disappears by the 15th cycle (*2, Fig. 3d). These changes
seem to be a strong predictor of the cell accepting \emph{less} charge
before the 4.2 V cut-off potential, evident from the respective
potential profiles and charging times. We note that the cell recovered
some capacity before continuing the trend toward degradation, and that
similar types of correlations between amplitude and capacity can be made
for other individual wave traces as well. While the changes are slight,
comparison of the overall recovered acoustic signal with the signal
recovered for specific ToF traces nevertheless reveals repeatable
correlations with SOH. This is intriguing as it enables the development
of detailed acoustic models in which hypothetical electrode degradation
mechanisms can be compared to the performance of a practical cell. The
development of such models is not a trivial endeavor and is currently
under investigation.

\textbf{Figure 3 \textbar{} Acoustic behavior of a
LiCoO\textsubscript{2}/graphite prismatic cell.} \textbf{(a,b)} ToF maps
for transmission and reflection modes, respectively, \textbf{(c)} total
reflected (red) and transmitted (green) signal amplitudes,
\textbf{(d,e)} traces for the amplitudes of transmitted waves 1 and 3,
respectively, \textbf{(f)} cell potential, and \textbf{(g)} applied
current as a function of cycling time. The vertical gray lines in panels
c-g represent transitions between charge, discharge, and rest steps;
arrows 1, 3, and 10, the dotted oval, as well as markings *1 and *2 are
discussed in the text.

{ }To evaluate the influence of cell construction and material
distribution on acoustic behavior, ultrasonic ToF analysis was performed
on a Li(NiCoAl)O\textsubscript{2} (NCA)/graphite 18650 ``jelly roll''
cell during galvanostatic cycling starting from a fresh state (selected
data shown in Fig. 4, full data set in Fig. S4). As shown in the ToF
maps (Fig. 4a and b), the acoustic behavior is significantly more
complicated in the 18650 cell than in the prismatic cell (Fig. 3). Based
on the simulated ToF maps from Fig. S1, this increase in acoustic
complexity is expected, as cells of this design typically consist of 15
to 25 layered windings.\textsuperscript{41} The total reflected and
transmitted signal intensities (Fig. 4c) as a function of SOC varies
significantly over the first 11 cycles, after which the acoustic
behavior stabilizes; this indicates the presence of an initial
``formation'' period, which is consistent with previous efforts on the
acoustic emission of batteries.\textsuperscript{25,27} As cycling
continues, the total transmitted signal at the end of charge becomes
increasingly large, as after \textasciitilde{}27 cycles new peaks
between 8 and 12 µs begin to appear in the transmission ToF map (Fig.
4a, indicated by the white arrow).

{ }New to the literature is a demonstration of the evolution of the
acoustic signal attenuation (as estimated from the fraction of the
interrogating (input) pulse measured by the transmission and reflection
transducers) of a lithium ion cell, as a function of cycle number.
Figure 5 shows a cycle-by-cycle comparison of the acoustic behavior of
the NCA/graphite 18650 cell, in which the total transmitted and
reflected signal amplitudes are separated by cycle number and
superimposed. There are a few interesting shifts, both at the end of
discharge and at the end of charge. In particular, attenuation of the
acoustic signal decreases significantly and consistently near the end of
discharge, as evidenced by the increase in reflected and transmitted
signal intensities. Most likely, this is a compliance change related to
the gradients of Li distribution within active cathode particles.
Diffusion limitations may lead to a scarcity or excess of lithium near
the surface of a particle, resulting in a dramatic change in its
mechanical properties (in a fashion similar to that described by
Woodford et. al.\textsuperscript{30}); this can either create local
disorder by increasing lattice mismatch with the more-lithiated regions
of the particle, leading to enhanced phonon scattering, or may simply
result in an increase in lattice stiffness. This effect is likely caused
by the cathode, as commercial lithium ion cells contain excess graphite
to prevent lithium plating during charge.\textsuperscript{41} During the
rest step the local lithium gradients relax, which in turn relaxes the
lattice strain, and the acoustic signal attenuation increases slightly
as a result. Attenuation of the acoustic signal becomes increasingly
strong at the end of charge with increasing cycle number. Similar to the
discharge step, this suggests dramatic changes in the mechanical
properties of electrodes, which is in agreement with static mechanical
analysis of electrodes.\textsuperscript{17--19} Similar trends exist in
the pouch cell after 50 cycles, but as the 18650 cell geometry is
significantly more complicated we hesitate to assert more than
correlations between acoustic behavior and states of charge and health.

\textbf{Figure 4 \textbar{} Acoustic behavior of an NCA/graphite 18650
cell from cycle 1 to 34. (a,b)} ToF maps for transmission and reflection
modes, respectively, \textbf{(c)} total reflected (red) and transmitted
(green) signal amplitudes, \textbf{(d)} cell potential, and \textbf{(e)}
applied current as a function of the cycling time. The vertical gray
lines in panels c-e represent transitions between charge, discharge, and
rest steps. The white arrow at \textasciitilde{}132 h in panel a is
discussed in the text.

\textbf{Figure 5 \textbar{}} Cycle-to-cycle analysis of the acoustic
behavior of an NCA/graphite 18650 cell. Evolution of (from top) the
transmitted and reflected signal amplitudes, cell potential, and applied
current as a function of cycling time for cycles 2 through 75 from the
acoustic/electrochemical data in Fig. 4. For each plot, the first cycle
is indicated in green, with subsequent cycles shown as progressively
darker shades of grey and the final cycle indicated in red.{~}

{ }While our computational model generally describes the experimentally
observed acoustic phenomena in lithium-ion batteries, there are notable
differences that demand further investigation. First, the 90˚ phase
shift between reflected and transmitted waves in the model is not seen
in either lithium-ion cell tested; we believe an additional layer may be
present in the cells that may ``rephase'' the signal, however a more
detailed consideration of individual layer effects is required. Second,
our model did not account for modulus changes within the electrodes
during cycling (though they are likely present), which probably
contributes to the more dramatic shifts in acoustic signal delay and
attenuation. Third, our model only considered density changes due to
lithium intercalation and de-intercalation; in reality, density and
modulus changes due to phase transformation, staging effects, etc. also
need to be considered. Fourth, due to the one-dimensional nature of our
model, heterogeneities in current density throughout the electrodes
during cycling were not accounted for. Furthermore, assumptions were
made regarding electrode porosity and acoustic homogeneity. Nonetheless,
the model presented herein does begin to describe the complex but
repeatable acoustic response within the batteries. This is an important
point that bears emphasizing: Though our model does not capture many of
the detailed acoustic and electrochemical processes, it nevertheless
captures much of the general behavior observed experimentally. This
clearly shows the feasibility of the electrochemical-acoustic approach
for SOC and SOC determination, and lays the groundwork for future
studies.

\textbf{Figure 6 \textbar{} Acoustic behavior of two brands of alkaline
AA cells. (a)} Transmission ToF map of an ultrasonic pulse in a Duracell
alkaline AA, with the corresponding cell potential and current profiles.
\textbf{(b)} Transmission ToF map of an ultrasonic pulse in a CVS brand
alkaline AA, with the corresponding cell potential and current profiles.
Both cells were discharged at 280 mA, corresponding to a C/10 rate.

{ }Ultrasonic time of flight experiments were also performed on
commercial Zn-MnO\textsubscript{2} alkaline AA cells from two different
manufacturers (Duracell and CVS) during galvanostatic discharge. As
shown in Fig. 6, the two alkaline cells exhibited similar overall
acoustic responses, however there are notable differences between them:
The transmitted signal of the Duracell was initially greater than that
of the CVS cell. We attribute this to the Duracell battery design,
specifically to the delamination of the Duralock\textsuperscript{TM}
corrosion protection layer. This proprietary polymeric coating on the Zn
anode particles increases mechanical contact between them and results in
the larger initial transmission signal compared to the CVS cell. When
the discharge current was applied, the acoustic signal attenuated in
both the Duracell and CVS brand batteries; this effect was more
pronounced in the Duracell (see Fig. S5 for more details). In both cells
we believe this initial attenuation is due in part to the nature of the
discharge reaction: in an alkaline (e.g. KOH) electrolyte, solid Zn is
stripped from the surface of individual particles and aqueous zincate
(Zn(OH)\textsubscript{4}\textsuperscript{2-}) ions form. This causes the
Zn particle network to become less packed, which in turn attenuates the
acoustic signal. In the Duracell, though, we also attribute some of the
acoustic attenuation to the disintegration of the
Duralock\textsuperscript{TM} coating. This is corroborated by
electrochemical impedance spectroscopy (EIS), shown in Fig. S6, in the
form of a large drop in total cell impedance after the onset of
discharge.

{ }As discharge continues in both cells, the transmission peak traces
between 8 and 20 µs undergo shifts in ToF that follow Nernstian-like
patterns, similar to the changes in cell potentials, and the acoustic
signal intensities are greatly enhanced by the end of discharge. This is
likely due to the formation of solid ZnO as the the saturation limit of
Zn(OH)\textsubscript{4}\textsuperscript{2-} is
reached.\textsuperscript{49,50} Eventually, near the end of discharge,
the ZnO creates a percolated network of ZnO within the anode, causing
the transmitted signal to increase dramatically.\textsuperscript{51} In
the Duracell, however, we note that after the current is applied the
signal intensity decreases until about halfway through the discharge
step before it begins to increase. This is distinct from the CVS cell,
in which (after the initial attenuation) the transmitted signal
increases monotonically throughout discharge. We believe this difference
is a result of the Duralock\textsuperscript{TM} corrosion protection
layer as well as proprietary electrolyte additives used by Duracell to
prevent corrosion and increase ZnO solubility. Indeed, upon dissection
of fresh batteries, we found Zn from the Duracell to be more lustrous
than that from the CVS cell, indicating less ZnO in the Duracell
initially.{~}

{ }There is also a set of ToF peaks between 4 and 8 µs that is more
pronounced in the Duracell battery than in the CVS battery. We attribute
this observation to differences in the cathode materials; the Duracell
cathode appears to have smaller, more densely packed particles than the
CVS cathode, as shown in confocal microscope images in Fig. S7, which
would result in a more mechanically-connected network. In both cells,
these peaks gradually fade in intensity (but do not shift in ToF) during
discharge. This is possibly due to the expulsion of aqueous electrolyte
from the anode to the cathode during discharge, which has been observed
via \emph{in-situ} neutron tomography by Riley et.
al.\textsuperscript{52} The increase in cathode water content would
cause the relevant transmission peaks to attenuate relative to a drier
cathode. After discharge, the relaxation behavior appears to be similar
for both cells. These results are particularly noteworthy because they
show that acoustic time of flight measurements can be sensitive enough
to detect differences in manufacturing processes between multiple brands
of the same battery chemistry.{~}

{ }To demonstrate the effect of discharge current on the acoustic
behavior of alkaline batteries, Duracells were discharged at different
rates. As shown in Fig. S8, the cells show a marked difference in their
acoustic transmission profiles as a function of discharge rate. The
morphology and spatial distribution of ZnO in alkaline batteries is
known to depend on discharge rate, as shown by Horn et
al.,\textsuperscript{53} so the different ToF profiles are probably due
to the influence of discharge rate on formation of ZnO and thus
mechanical properties of the full cell. Higher discharge rates result in
ZnO forming predominantly in the regions closest to the anode/separator
interface, while lower discharge rates result in a more uniform
distribution of ZnO formation throughout the anode. Thus, cells that are
discharged more slowly develop a ZnO network that is more percolated
(and thus transmits sound more readily) than cells that are discharged
more quickly.\textsuperscript{51} A detailed understanding of the
structural and mechanical changes that are responsible for the ToF
behavior in alkaline cells will be addressed further in a later
publication.

Conclusions

{ }Electrochemical-acoustic time-of-flight experiments were performed on
several batteries, including a LiCoO\textsubscript{2}/graphite pouch
cell, a cylindrical Li(NiCoAl)O\textsubscript{2}/graphite 18650 cell,
and on two types of Zn/MnO\textsubscript{2} alkaline AA cells. Our
results demonstrate strong correlations between SOC and the density
distribution within a cell, as determined by the acoustic measurements,
and suggest that this is an effective analysis technique regardless of
battery chemistry and form factor. Beyond SOC, the changes in density
are indicative of underlying physical processes occurring in the
electrode materials during cycling, such as the ``formation'' period in
the as-received LiNCA/graphite 18650 cell as well as the degradation of
the LiCoO\textsubscript{2}/graphite pouch cell. Changes in the ToF echo
profiles and acoustic signal amplitudes as a function of cycle number
appear to be key indicators of critical phenomena occurring within the
battery, including changes in intraparticle and interparticle stress and
strain, as well as the formation and removal of critical surface layers
(SEI/passivation, artificial and natural). Such correlations suggest
that the electrochemical-acoustic data can be used to determine SOH.

We have shown through acoustic modeling that the key constituents of
sound speed, i.e., bulk modulus and density, are closely related to the
observed changes in the measured acoustic signal during electrochemical
cycling. These properties, to date, have been exceptionally difficult to
determine \emph{in situ} or \emph{in operando}. It is important to note
that our model did not include many of the non-linear physical process
that are known to occur during cycling. Furthermore, commercial cells
typically contain many more layers than were included in our
simulations. As such, the model that we presented is not accurate enough
for layer-by-layer decoupling of the experimental ToF data, nor can it
be used as a predictive tool (the development of such models is not a
trivial endeavor, as it requires exact layer-by-layer control when
constructing cells in the lab). Nevertheless, our model was able to
qualitatively capture much of the acoustic behavior that was observed
experimentally. This is a clear indication of the feasibility of our
electrochemical-acoustic approach, and the work presented herein lays
the groundwork for future studies to refine the details of the model and
our understanding of the technique.

The electrochemical acoustic ToF method is powerful because it provides
a fingerprint of a battery's chemistry and geometr. While a more
detailed model is required to decouple the individual effects, this is a
tool that is comparable to EIS in insight and utility, where a
well-designed model in conjunction with experimental data provides
fundamental insights. The quality of the acoustic data is complemented
by the universal applicability of this approach. While ultrasonic ToF
analysis cannot directly probe chemical and structural properties in the
same way that neutron, x-ray and electron methods can, it is still able
to provide physical insights that are not possible with standard
electrochemical equipment at a fraction of the cost of photonic
interrogation methods. Unlike EIS and other inline electrochemical test
methods, acoustic ToF analysis can be done without electrical contact
and even with only one point of physical contact if operating in
pulse/echo mode. Furthermore, unlike photonic characterization, this
method can be applied readily to commercial off-the-shelf cells of very
different chemistries and form factors.{~}

This simple, high-speed technique that provides heretofore unmeasurable
physical correlations for large-scale complex batteries, as well as
physical insights that have only previously been available through
high-energy x-ray analysis. The technique is effective across all
battery technologies because it exploits a common thread to all cells:
critical manufacturing control of layers and the shifting of mass within
the cell. While different batteries will exhibit different acoustic
progressions during operation, the same theory should be broadly
applicable. In this article we demonstrated new physical insights to a
substantial majority of the worlds batteries, namely the nearly 2
billion cells manufactured in 2014 considering both AA Alkaline and
18650 Li-ion batteries.\textsuperscript{54--56} To date, ultrasonic
methods for battery analysis have been limited to careful laboratory
studies or analysis of dramatic failure, and we have shown here that
they can be used broadly to learn and probe much more about all types of
cells. The methods can be applied to batteries during operation, using
equipment that can be readily integrated into devices as small and
simple as a cell phone (e.g. basic signal processing electronics and
simple piezoelectric transducers). This provides rich opportunities for
multiplex testing within a lab environment as well as field analysis of
the physical properties of batteries, all without any modification to
the cells themselves.

Acknowledgements

{ }The authors would like to thank Professor J. Sakamoto at University
of Michigan for discussion and suggestions on hardware. The authors
would also like to thank J. Canarella and the Arnold Group at Princeton
University for providing us with aged prismatic pouch cells as well as
enlightening discussion. A.G. would like to thank the Swiss National
Science Foundation for the Early Postdoc Mobility Grant.

{ }This work was supported by DOE ARPA-E RANGE DE-AR0000400, NSF CMMI
1402872, the Princeton Project X Fund, and the Swiss National Science
Foundation.

Additional Information

\textsuperscript{a}Department of Mechanical and Aerospace Engineering,
Princeton University, Princeton, New Jersey 08544, United States

\textsuperscript{b}Department of Electrical Engineering Princeton
University, Princeton, New Jersey 08544, United States

\textsuperscript{c}Andlinger Center for Energy and the Environment,
Princeton University, Princeton, New Jersey 08544, United States

*Correspondence and request for materials should be directed towards
D.S. email: steingart@princeton.edu

Competing Financial Interests

The authors declare no competing financial interests.

References

1.{ }J. W. Gallaway, D. Desai, A. Gaikwad, C. Corredor, S. Banerjee, and
D. Steingart, \emph{J. Electrochem. Soc.}, 2010, \textbf{157},
A1279--A1286.

2.{ }Y. Ito, M. Nyce, R. Plivelich, M. Klein, and D. Steingart, \emph{J.
Power Sources}, 2011, \textbf{196}, 2340--2345.

3.{ }A. Radisic, P. M. Vereecken, J. B. Hannon, P. C. Searson, and F. M.
Ross, \emph{Nano Lett.}, 2006, \textbf{6}, 238--242.

4.{ }K. Karki, G. G. Amatucci, M. S. Whittingham, and F. Wang,
\emph{Microsc. Microanal.}, 2014, \textbf{20}, 512--513.

5.{ }D. Santhanagopalan, D. Qian, T. McGilvray, Z. Wang, F. Wang, F.
Camino, J. Graetz, N. Dudney, and Y. S. Meng, \emph{J. Phys. Chem.
Lett.}, 2014, \textbf{5}, 298--303.

6.{ }F. Marone and M. Stampanoni, \emph{Adv. Technol. Electr. Eng.
Energy}, 2013, \textbf{3}, 845--850.

7.{ }X. Liu, D. Wang, G. Liu, V. Srinivasan, Z. Liu, Z. Hussain, and W.
Yang, \emph{Nat. Commun.}, 2013, \textbf{4}, 2568.

8.{ }Y.-C. K. Chen-Wiegart, J. Wang, and J. Wang, in \emph{SPIE Optical
Engineering + Applications}, International Society for Optics and
Photonics, 2013, p. 88510C--88510C--5.

9.{ }J. Nelson, S. Misra, Y. Yang, A. Jackson, Y. Liu, H. Wang, H. Dai,
J. C. Andrews, Y. Cui, and M. F. Toney, \emph{J. Am. Chem. Soc.}, 2012,
\textbf{134}, 6337--6343.

10.{ }G. Du, N. Sharma, and V. K. Peterson, \emph{Adv. Funct. Mater.},
2011, \textbf{21}, 3990--3997.

11.{ }I. Manke, J. Banhart, A. Haibel, A. Rack, S. Zabler, N. Kardjilov,
A. Hilger, A. Melzer, and H. Riesemeier, \emph{Appl. Phys. Lett.}, 2007,
\textbf{90}, 214102.

12.{ }A. R. Armstrong, M. Holzapfel, and P. Novák, \emph{Journal of the
American Chemical Society}, 2006, \textbf{128}, 8694--8698.

13.{ }J. W. Gallaway, C. K. Erdonmez, Z. Zhong, M. Croft, L. A.
Sviridov, T. Z. Sholklapper, D. E. Turney, S. Banerjee, and D. A.
Steingart, \emph{J. Mater. Chem. A Mater. Energy Sustain.}, 2014,
\textbf{2}, 2757--2764.

14.{ }J. W. Gallaway, M. Menard, B. Hertzberg, Z. Zhong, M. Croft, L. A.
Sviridov, D. E. Turney, S. Banerjee, D. A. Steingart, and C. K.
Erdonmez, \emph{J. Electrochem. Soc.}, 2015, \textbf{162}, A162--A168.

15.{ }J. Rijssenbeek, Y. Gao, Z. Zhong, M. Croft, N. Jisrawi, A.
Ignatov, and T. Tsakalakos, \emph{J. Power Sources}, 2011, \textbf{196},
2332--2339.

16.{ }A. Senyshyn, M. J. Mühlbauer, K. Nikolowski, T. Pirling, and H.
Ehrenberg, \emph{J. Power Sources}, 2012, \textbf{203}, 126--129.

17.{ }J. Cannarella and C. B. Arnold, \emph{J. Power Sources}, 2014,
\textbf{245}, 745--751.

18.{ }J. Cannarella and C. B. Arnold, \emph{J. Power Sources}, 2013,
\textbf{226}, 149--155.

19.{ }J. Cannarella and C. B. Arnold, \emph{J. Power Sources}, 2014,
\textbf{269}, 7--14.

20.{ }G. Y. G. C. J, J. H. Prévost, and C. B. Arnold, \emph{Journal of
The Electrochemical Society}, 2014, \textbf{161}, F3065--F3071.

21.{ }R. Deshpande, M. Verbrugge, Y.-T. Cheng, J. Wang, and P. Liu,
\emph{J. Electrochem. Soc.}, 2012, \textbf{159}, A1730--A1738.

22.{ }Koyama Y~ Chin T E Rhyner, R. K. Holman, and S. R. Hall,
\emph{Adv. Funct. Mater.}, 2006, \textbf{16}, 492--498.

23.{ }S. Didier-Laurent, H. Idrissi, and L. Roué, \emph{J. Power
Sources}, 2008, \textbf{179}, 412--416.

24.{ }A. Etiemble, H. Idrissi, S. Meille, and L. Roué, \emph{J. Power
Sources}, 2012, \textbf{205}, 500--505.

25.{ }S. Komagata, N. Kuwata, R. Baskaran, J. Kawamura, K. Sato, and J.
Mizusaki, \emph{ECS Trans.}, 2010, \textbf{25}, 163--167.

26.{ }N. Kircheva, S. Genies, D. Brun-Buisson, and P.-X. Thivel,
\emph{J. Electrochem. Soc.}, 2011, \textbf{159}, A18--A25.

27.{ }K. Rhodes, M. Kirkham, R. Meisner, C. M. Parish, N. Dudney, and C.
Daniel, \emph{Rev. Sci. Instrum.}, 2011, \textbf{82}, 075107.

28.{ }K. Rhodes, N. Dudney, E. Lara-Curzio, and C. Daniel, \emph{J.
Electrochem. Soc.}, 2010, \textbf{157}, A1354--A1360.

29.{ }W. H. Woodford, Y.-M. Chiang, and W. C. Carter, \emph{J. Mech.
Phys. Solids}, 2014, \textbf{70}, 71--83.

30.{ }W. H. Woodford, W. Craig Carter, and Y.-M. Chiang, \emph{Energy
Environ. Sci.}, 2012, \textbf{5}, 8014--8024.

31.{ }W. H. Woodford, Y.-M. Chiang, and W. C. Carter, \emph{J.
Electrochem. Soc.}, 2010, \textbf{157}, A1052--A1059.

32.{ }W. H. Woodford, W. C. Carter, and Y.-M. Chiang, \emph{J.
Electrochem. Soc.}, 2014, \textbf{161}, F3005--F3009.

33.{ }B. Sood, M. Osterman, and M. Pecht, in \emph{Product Compliance
Engineering (ISPCE), 2013 IEEE Symposium on}, 2013, pp. 1--6.

34.{ }B. Sood, C. Hendricks, M. Osterman, and M. Pecht, in
\emph{International Symposium for Testing and Failure Analysis (ISTFA)},
2013, pp. 4--16.

35.{ }\emph{Non-Contact Ultrasonic Evaluation of Prismatic Lithium Ion
Battery Cells}, The Ultran Group, 2012.

36.{ }E. S. Furgason, V. L. Newhouse, N. M. Bilgutay, and G. R. Cooper,
\emph{Ultrasonics}, 1975, \textbf{13}, 11--17.

37.{ }B. Banks, \emph{Ultrasonics}, 1963, \textbf{1}, I.

38.{ }A. Vary, \emph{Acousto-ultrasonic characterization of fiber
reinforced composites}, NASA, 1982.

39.{ }R. LeVeque, \emph{Clawpack}.

40.{ }J. N. Reimers and J. R. Dahn, \emph{J. Electrochem. Soc.}, 1992,
\textbf{139}, 2091--2097.

41.{ }T. Reddy and J. Linden, \emph{Handbook of Batteries}, McGraw-Hill,
2004, vol. III.

42.{ }F. T. Schulitz, Y. Lu, and H. N. G. Wadley, \emph{J. Acoust. Soc.
Am.}, 1998, \textbf{103}, 1361--1369.

43.{ }D. J. McClements, \emph{Adv. Colloid Interface Sci.}, 1991,
\textbf{37}, 33--72.

44.{ }R. E. Challis, M. J. W. Povey, M. L. Mather, and A. K. Holmes,
\emph{Rep. Prog. Phys.}, 2005, \textbf{68}, 1541.

45.{ }P. Albertus and J. Newman, 2007.

46.{ }D. Steingart, \emph{GitHub}.

47.{ }J. D. N. Cheeke, \emph{Fundamentals and applications of ultrasonic
waves}, CRC press, 2012.

48.{ }A. Van der Ven, M. K. Aydinol, G. Ceder, G. Kresse, and J. Hafner,
\emph{Phys. Rev. B Condens. Matter}, 1998, \textbf{58}, 2975--2987.

49.{ }T. P. Dirkse, C. Postmus, and R. Vandenbosch, \emph{J. Am. Chem.
Soc.}, 1954, \textbf{76}, 6022--6024.

50.{ }S. Szpak and C. J. Gabriel, \emph{J. Electrochem. Soc.}, 1979,
\textbf{126}, 1914--1923.

51.{ }S. Bhadra, B. J. Hertzberg, A. G. Hsieh, M. Croft, J. W. Gallaway,
B. J. Van Tassell, M. Chamoun, C. Erdonmez, Z. Zhong, T. Sholklapper,
and D. A. Steingart, \emph{J. Mater. Chem. A}, 2015 (\emph{in press})

52.{ }G. V. Riley, D. S. Hussey, and D. Jacobson, \emph{ECS Trans.},
2010, \textbf{25}, 75--83.

53.{ }Q. C. Horn and Y. Shao-Horn, \emph{J. Electrochem. Soc.}, 2003,
\textbf{150}, A652--A658.

54.{ }B. Carter, \emph{Battery Manufacturing in the US Market Research},
IBISworld, 2014.

55.{ }\emph{Primary Batteries, Dry and Wet}, Highbeam Business, 2009.

56.{ }M. Anderman, \emph{The Tesla Battery Report 2014 - Overview -
Advanced Automotive Batteries}, Advanced Automotive Batteries, 2014.

\end{document}
