\section{Conclusions}
\label{sec:bw:conclusion}

This chapter demonstrates that electrochemical-acoustic time-of-flight experiments were performed on several batteries, including a \ce{LiCoO2}/graphite pouch cell, a cylindrical~\ce{Li(NiCoAl)O2}/graphite 18650 cell, and on two types of Zn/~\ce{MnO2} alkaline AA cells demonstrate strong correlations between SOC and the density distribution within a cell, and suggest that this is an effective analysis technique regardless of battery chemistry and form factor. Beyond SOC, the changes in density are indicative of underlying physical processes occurring in the electrode materials during cycling, such as the “formation” period in the as-received LiNCA/graphite 18650 cell as well as the degradation of the~\ce{LiCoO2}/graphite pouch cell. Changes in the ToF echo profiles and acoustic signal amplitudes as a function of cycle number appear to be key indicators of critical phenomena occurring within the battery, including changes in intraparticle and interparticle stress and strain, as well as the formation and removal of critical surface layers (SEI/passivation, artificial and natural). Such correlations suggest that the electrochemical-acoustic data can be used to determine SOH.
	
This chapter has also shown through acoustic modeling that the key constituents of sound speed, i.e., bulk modulus and density, are closely related to the observed changes in the measured acoustic signal during electrochemical cycling. These properties, to date, have been exceptionally difficult to determine in situ or in operando. It is important to note that our model did not include many of the non-linear physical process that are known to occur during cycling. Furthermore, commercial cells typically contain many more layers than were included in our simulations. As such, the model that we presented is not accurate enough for layer-by-layer decoupling of the experimental ToF data, nor can it be used as a predictive tool (the development of such models is not a trivial endeavor, as it requires exact layer-by-layer control when constructing cells in the lab). Nevertheless, our model was able to qualitatively capture much of the acoustic behavior that was observed experimentally. This is a clear indication of the feasibility of our electrochemical-acoustic approach, and the work presented herein lays the groundwork for future studies to refine the details of the model and our understanding of the technique.

The electrochemical acoustic ToF method is powerful because it provides a fingerprint of a battery’s chemistry and geometry. While a more detailed model is required to decouple the individual effects, this is a tool that is comparable to EIS in insight and utility, where a well-designed model in conjunction with experimental data provides fundamental insights. The quality of the acoustic data is complemented by the universal applicability of this approach. While ultrasonic ToF analysis cannot directly probe chemical and structural properties in the same way that neutron, x-ray and electron methods can, it is still able to provide physical insights that are not possible with standard electrochemical equipment at a fraction of the cost of photonic interrogation methods. Unlike EIS and other inline electrochemical test methods, acoustic ToF analysis can be done without electrical contact and even with only one point of physical contact if operating in pulse/echo mode. Furthermore, unlike photonic characterization, this method can be applied readily to commercial off-the-shelf cells of very different chemistries and form factors. 

This simple, high-speed technique that provides heretofore unmeasurable physical correlations for large-scale complex batteries, as well as physical insights that have only previously been available through high-energy x-ray analysis. The technique is effective across all battery technologies because it exploits a common thread to all cells: critical manufacturing control of layers and the shifting of mass within the cell. While different batteries will exhibit different acoustic progressions during operation, the same theory should be broadly applicable.